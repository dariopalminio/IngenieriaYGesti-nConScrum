\chapter{Filosofía Scrum}

Discriminando todas las filosofías generales, como ideologías y concepciones religiosas, podemos decir que hay filosofías relacionadas específicamente al desarrollo de sistemas y de software. Pues, si bien ocurre que ideologías políticas influyen en los desarrolladores, líderes y arquitectos hay ideas más a fines del ámbito de desarrollo de sistemas, ideas que podemos decir que conforman filosofías influenciadoras. Estas influencias son notorias cuando vemos que se toma una decisión de usar una determinada metodología que implementa determinados principios o seguir determinados principios sin datos empíricos que sostienen su uso ni explicaciones racionales soportadas con evidencia. A veces dichos principios son como sacados de la galera o reflejan expresión de deseos. Tal como sucede cuando se aplican determinadas tecnologías o metodologías como si se tratasen de una moda o de algo aparentemente arbitrario. También esto se puede apreciar cuando escuchamos decir en una reunión de trabajo en equipo que se premiará al individuo que sobresale (filosofía individualista), que “el éxito del grupo está por encima del individual” (filosofía Ubuntu o cooperativa), escuchamos en una reunión técnica que “el software debe ser libre” (filosofía Free Software) o que solo se desarrollara software propietario, que “las mejores arquitecturas, requisitos y diseños emergen de equipos autoorganizados” \cite{Beck-2001} (filosofía ágil); o que “el software es un mundo de objetos” (filosofía del Paradigma Orientado a Objetos). Las consignas o lemas organizacionales, los mantras personales o institucionales, las visiones empresariales y lineamientos institucionales también son, en muchos casos, una expresión de filosofías que condicionan a las personas y al desarrollo de software. Por ejemplo, en particular, se da con las filosofías del pensamiento sistémico, pensamiento crítico, filosofía ágil, software libre, software abierto o software propietario, que son las principales que en el mundo del desarrollo de software influencian a sus actores. Estas filosofías suelen presentar principios a seguir y estos principio guían algunas metodologías (prácticas y métodos) de desarrollo de sistemas y a gran parte de desarrolladores de sistemas. Aquí nos vamos a centrar en la filosofía Scrum y la filosofía Ágil.

\section{Mentalidad y modelos mentales}

Las personas que participan en el desarrollo de sistemas, diseños organizacionales o industrias de software tienen experiencias, creencias, principios, vivencias y valores que repercuten y condicionan el modo en que ellas perciben la realidad de su día a día y, en consecuencia, repercuten y condicionan su actuar, su forma de hacer, su trabajo y el resultado del mismo, sistemas hombre-máquina o software. Estas ideas, en los gerentes y desarrolladores, son modelos mentales que conforman una mentalidad, o lo que se denomina mindset (Masa Maeda 2012). O sea que la filosofía que una persona siga o adhiera está relacionada a los modelos mentales de esa persona y forja su mentalidad o mindset que en algún sentido lo guía.

Los modelos mentales pueden definirse como: “imágenes internas, que están profundamente arraigadas, de cómo funciona el mundo, imágenes que nos limitan a las formas familiares de pensar y actuar”. Los modelos mentales abarcan cuestiones acerca de cómo vemos el mundo y de cómo actuamos en el mundo. 

Tener noción de los modelos mentales es importante para entender que hay detras de las acciones, detras de las prácticas y de las metodologías usadas. Nos ayuda a comprender que para realizar cambios en las acciones de las personas y cambios en la forma de hacer las cosas es necesario, la mayoría de las veces, cambiar la mentalidad. Sin cambio de mentalidad puede hacerse incostenible en el tiempo un cambio de hábito, práctica o metodología. Por ese motivo, seguir principios en forma mecánica o por ovediencia a la autoridad no forma convicción, y seguir una filosofía sin convicción es un primer condicionante de fracaso práctico en su implementación.

\section{Principios}

En la industria de sistemas y software los principios son reglas, proposiciones o normas que funcionan como máximas o preceptos que orientan la acción. O sea que un principio es como “una regla general de conducta o comportamiento” \cite{Lawson-Martin-2008} \cite{SEBoK-2014}. Un principio es como una recomendación sin el cómo se sigue la recomendación. Cómo hará para seguir la recomendación depende de usted o de quien decida seguir el principio. Por eso son la base, origen y razón fundamental sobre la cual se procede o discurre en materia de sistemas. Se suele usar en el contexto de procesos de desarrollo y metodologías como proposición que da razón, punto de partida o guía, como fundamento de un conjunto de prácticas, una metodología o paradigma de trabajo siendo las metodologías las encargadas de definir el cómo se implementan.

Cada uno sigue algunos principios en su vida como: "trabajo colaborando". También seguimos principios éticos como: “nunca mentir” o “no robar”. A semejanza de los principios éticos tradicionales, ocurre que los principios no necesariamente tienen fundamento objetivo, racional, empírico o basado en evidencias; pues, en ocasiones se comportan más como principios filosóficos que como principios científicos. Lo cual no quiere decir que no deba buscarse que los principios en ingeniería no sean sacados de la galera o usados de forma irracional, sin justificativo razonable y sin comprobación. Es preferible aplicar principios de comprobada efectividad en su aplicación. El aplicar principios comprobados reduce la cantidad de tiempo necesaria para crear las salidas de planificación de los recursos humanos y mejora la probabilidad de que la planificación sea efectiva \cite{PMBOK-2004}, del mismo modo aplicar principios comprobados en actividades de desarrollo y diseño de sistemas reduce tiempos de investigación para generar la salida deseada y minimiza riesgos, mejorando la probabilidad de que el desarrollo sea efectivo. 

Los principios de Scrum son las pautas básicas para aplicar el marco de Scrum y guía a usarse en todos los proyectos Scrum \cite{SBOK-2013}, proyectos en los que se aplica metodología Scrum. Los principios de Scrum se orientan a la gestión de proyectos, desarrollo de productos, trabajo en equipo y el trabajo en base a los principios ágiles. O sea que los principios Scrum tienen correlación con los principios ágiles en forma prácticamente directa \cite{Agile-Atlas-2012}. Y los principios de Scrum están alineados a los valores de Scrum que son: Foco, Coraje, Apertura, Compromiso y Respeto.


A continuación se describen los principios y valores del Manifieto Ágil y de Scrum.

\section{Manifiesto Ágil}

Los principios del desarrollo ágil se encuentran en el Manifiesto por el Desarrollo Ágil de Software o Manifiesto Ágil [Manifesto Agile 2001], el que expone valores y principios firmado por diecisiete personas convocadas por Kent Beck \cite{Beck-2001}. Los principios del desarrollo ágil surgieron como principios base y originarios de métodos que estaban surgiendo como alternativa a las metodologías clásica formales (CMMI, SPICE, etc.) a las que, autores como Kent Beck, consideraban excesivamente pesadas y rígidas por su carácter normativo y fuerte dependencia de planificaciones detalladas completas y previas al desarrollo \cite{Wiki-2015}.

\subsection{Valores del Manifiesto Ágil}

El Manifiesto Ágil propone los siguientes valores:

\begin{enumerate}

\item \textbf{Individuos e interacciones sobre procesos y herramientas:} hay que priorizar la confianza puesta en los equipos, los individuos dentro de esos equipos y la manera en que éstos interactúan en vez de seguir rígidamente procesos y herramientas. Pues, son los equipos quienes deben resolver qué hay que hacer, cómo hay que hacerlo y finalmente son ellos quienes lo hacen. Pues los equipos no deberán ser meros autómatas que reciben órdenes jerárquicas de la organización de arriba hacia abajo sino que, en lugar de eso, se espera que de abajo hacia arriba sepan resolver los problemas, ofrecer sus propios métodos de trabajo y ser hasta cierto punto autosuficientes. En este sentido, hay que delegar en ellos la identificación de qué se interpone en el camino de sus metas y la asunción de la responsabilidad de resolver todas las dificultades que se encuentren dentro de su alcance. Se les debe permitir trabajar en conjunto con otras partes de la organización para resolver asuntos que están más allá de su control.

\item \textbf{Software funcionando sobre documentación extensiva:} hay que estar orientado al producto o focalizarse en él y de este modo requerir un incremento de producto completo y funcionando como resultado final de cada ciclo de trabajo, en vez de tener que cumplir con grandes y engorrosas documentaciones y formalismos burocráticos. Ciertamente, en la construcción del producto, sea necesario realizar determinada documentación, pero es el producto concreto o funcionando lo que permite a la organización guiar al proyecto hacia el éxito. Es crucial que los equipos produzcan un incremento de producto en cada ciclo de trabajo.

\item \textbf{Colaboración con el cliente sobre negociación contractual:} en vez de tener comunicación pobre debido a restricciones contractuales, el cliente debería ser el punto de contacto principal del equipo, en colaboración de trabajo, con los eventuales usuarios finales del producto y con las partes de la organización que necesitan el producto. Se debería ver al cliente como un miembro del equipo que trabaja colaborativamente con el resto de integrantes para decidir qué debe hacerse y qué no. Con el cliente se debería poder seleccionar el trabajo que debe realizarse a continuación, asegurando que el producto tenga el valor más alto posible en todo momento. Esto es crucial construir una fuerte colaboración con el cliente en vez de negociar contratos rígidos que obstaculizan el trabajo ágil y generan fricción con el cliente.

\item \textbf{Respuesta ante el cambio sobre seguir un plan:} el avance del trabajo o del equipo debería estar representado por un incremento de producto real y que funciona, y no por una correcta correlación y contrastación a un plan. Se debe priorizar la flexibilidad para adaptación a cambios en vez de la rigidez de seguir un plan detallado. Para ello, los equipos deberían inspeccionar lo que sucede de forma abierta y transparente buscando adaptar sus acciones a la realidad. O sea que, la planificación se debería adaptar a la realidad y al equipo y no al revés.

\end{enumerate}

Se puede notar que estos valores son aplicables a cualquier tipo de organización e industria. Pues, solo el segundo valor se refiere particularmente a la industria de software y el mismo se puede reformular de forma más general como sigue: trabajar orientados al producto funcionando más que sobre una amplia y extensa documentación \cite{Sriram-Narayan-2015}.

\subsection{Principios del Manifiesto Ágil}

Alineados a estos valores, el Manifiesto Ágil propone los siguientes principio:

\begin{enumerate}

\item \textbf{Entregar valor temprano y continua:} Nuestra mayor prioridad es satisfacer al cliente mediante la entrega temprana y continua de software con valor (alineado al principio de "entregar lo más rápido posible" de Lean).

\item \textbf{Apertura al cambio:} Aceptamos que los requisitos cambien, incluso en etapas tardías del desarrollo. Los procesos Ágiles aprovechan el cambio para proporcionar ventaja competitiva al cliente.

\item \textbf{Entregables frecuentes:} Entregamos software funcional frecuentemente, entre dos semanas y dos meses, con preferencia al periodo de tiempo más corto posible (alineado al principio de "entregar lo más rápido posible" de Lean).

\item \textbf{Cooperación con el cliente:} Los responsables de negocio y los desarrolladores trabajamos juntos de forma cotidiana durante todo el proyecto (alineado al principio de "potenciar al equipo" de Lean). O sea que se privilegia una cooperación constante entre miembros del equipo e interesados externos al equipo (como clientes).

\item \textbf{Personas motivadas:} Los proyectos se desarrollan en torno a individuos motivados. Hay que darles el entorno y el apoyo que necesitan, y confiarles la ejecución del trabajo (alineado al principio de "entregar lo más rápido posible" de Lean).

\item \textbf{Conversación cara a cara:} El método más eficiente y efectivo de comunicar información al equipo de desarrollo y entre sus miembros es la conversación cara a cara.

\item \textbf{Orientación al producto:} El software funcionando es la medida principal de progreso.

\item \textbf{Desarrollo sostenible:} Los procesos Ágiles promueven el desarrollo sostenible. Los promotores, desarrolladores y usuarios debemos ser capaces de mantener un ritmo constante de forma indefinida.

\item \textbf{Excelencia técnica:} La atención continua a la excelencia técnica y al buen diseño mejora la Agilidad.

\item \textbf{Simplicidad eficiente:} La simplicidad, o el arte de maximizar la cantidad de trabajo no realizado, es esencial (alineado al principio de simplicidad). Este estilo de prácticas es parte de algo que, en metodologías ágiles, se conoce como: "hacer todo lo posible por hacer lo menos posible" \cite{Anacleto-2005}. Por ejemplo en Arquitectura se puede considerar mantener una lo más simple posible. Si la simpleza se traduce en facilidad de uso de la arquitectura, facilidad para entender los conceptos involucrados y documentación necesaria, es muy probable que el nivel de productividad aumente \cite{Anacleto-2005}. En este sentido la simpleza de la arquitectura es un requerimiento de calidad alineado a la filosofía ágil.

\item \textbf{Equipos auto-organizados:} Las mejores arquitecturas, requisitos y diseños emergen de equipos auto-organizados (alineado al principio sistémico de emergencia).

\item \textbf{Equipos auto-reflexivos:} A intervalos regulares el equipo reflexiona sobre cómo ser más efectivo para a continuación ajustar y perfeccionar su comportamiento en consecuencia.

\end{enumerate}

\section{Valores de Scrum}

\begin{enumerate}
\item \textbf{Foco:} hay que enfocarse en sólo unas pocas cosas a la vez, trabajamos bien juntos y buscando producir un resultado excelente tratando de entregar ítems valiosos en forma pronta.

\item \textbf{Coraje:} hay que buscar sentirse apoyados y tener más recursos a disposición para promover el coraje para enfrentar desafíos más grandes. Además, para poder lograr cambios significativos en una organización que mantiene una cultura con principios y valores que entran en conflicto con los de Scrum y el Manifiesto Ágil, es necesario tener coraje para impulsar el cambio y plantarse en forma efectiva ante la resistencia al cambio. En este sentido, coraje se refiere al valor que hay que tener para no dejarse dominar ante la idiosincrasia predominante y el “status quo” que atentan contra el pensamiento Scrum.

\item \textbf{Apertura:} Hay que tener apertura para expresar cotidianamente cómo nos va, qué problemas encontramos, manifestar las preocupaciones y aceptar las sugerencias de los pares para que éstas puedan ser tomadas en cuenta por nosotros y por los demás. La apertura requiere capacidad de aceptación y tolerancia ante la crítica y la opinión de los demás.

\item \textbf{Compromiso:} Se busca lograr compromiso para el éxito gracias a promover el mayor control nuestro sobre lo que hacemos y nuestro destino. Hay mayor probabilidad de lograr compromiso en las personas que deciden sobre lo que hacen.

\item \textbf{Respeto:} Buscamos convertirnos en merecedores de respeto a medida que trabajamos juntos, compartiendo éxitos y fracasos, llegando a respetarnos los unos a los otros y ayudándonos mutuamente.
\end{enumerate}

\section{Principios de Scrum}

Se suele asociar a los valores del Manifiesto Ágil como principios de Scrum. Por lo que, en primera instancia, los cuatro valores ágiles son los principales principios de Scrum. Pero para no repetirlos en esta sección vamos a nombrar otros seis principios que en diferente bibliografía se suelen atribuir a Scrum. Los mismos son:

\begin{enumerate}

\item \textbf{Control Empírico:} El proceso empírico de control del proyecto es más efectivo que el control predictivo de largos plazos. Es más efectivo para gestionar la complejidad y obtener el mayor valor posible, basado en inspección y adaptación regular en función de los resultados que se van obteniendo y del propio contexto del proyecto. El proceso empírico permite adaptabilidad a requisitos que emergen del mismo proceso de desarrollo. Con este principio como marco, podemos usar metodologías, prácticas y técnicas, y ser nosotros (o el propio equipo de trabajo) los que a través del empirismo, determinemos la forma más adecuada de hacer las cosas para lograr los objetivos. Son los equipos de desarrollo los que deben hacer lo que sea necesario para entregar el producto esperado y aprender de su propia experiencia mediante exploración y experimentación \cite{UNTREF-2014}. Es el equipo el que determina qué prácticas y herramientas les dan los mejores resultados, y así mejoran de manera continua. Los buenos equipos trabajarán constantemente en mejorar y aprender de sus experiencia. Además, se debe aprender de la experiencia de los demás leyendo libros y buscando la experiencia de personas que ya hayan venido probando algunas prácticas, o que están experimentando con nuevas potenciales mejores formas de hacer las cosas a través de la inspección y la adaptación. 

\item \textbf{Auto-organización:} Los equipos auto-organizados pueden auto-gestionarse y de ellos emerge la sabiduría necesaria para la gestión de sus proyectos y actividades, y así lograr la sinergia necesaria para resolver problemas en forma ágil. Esta idea proviene de la concepción de que de un sistema social puede emerger inteligencia de grupo como puede suceder en una bandada de pájaros. Esta es una premisa aceptada en Inteligencia Artificial, pensamiento sistémico y en filosofía emergentista. Se considera que en un sistema con agentes inteligentes, a partir de reglas locales simples puede emerger inteligencia grupal colectiva o de enjambre, pues se considera que la "información local puede conducir a la sabiduría global" \cite{Steven-Johnson-2002}. De aquí que, de la auto-organización en equipos de trabajo puede surgir inteligencia o también sabiduría según un fenómeno conocido como "sabiduría de multitud" o "wisdom of the crowd" \cite{MIT-Press-2009} en la que la opinión colectiva de un grupo puede ser mejor que la individual de un experto. Este fenómeno, no sólo es útil a la gestión de proyectos, sino también a las actividades de estimación, pues el promedio de muchas estimaciones individuales suele estar mucho más cerca del valor real que la estimación de un experto. En lo referente al diseño se cree, como lo indica el principio 11 (once) del Manifiesto Ágil, que se logran mejores diseños y arquitecturas desde equipos auto-organizados \cite{UNTREF-2014}. En lo referente al liderazgo se cree que no es necesario un líder jerárquico, autoritario o experto que guíe al equipo sino que es el propio equipo el que genera su liderazgo. Según esta perspectiva se puede prescindir de la figura de líder tradicional o jefe, se puede carecer de líder, lograr muchos líderes o tener un líder con perfil más bien de facilitador (servicial e integrador).

\item \textbf{Colaboración:} Se puede entender a la colaboración como la capacidad de reconcebir nuestras propias ideas a la luz de la de las demás \cite{Austin-2003} para poder lograr ideas colectivas mejores que las ideas individuales \cite{UNTREF-2014}. Las ideas en colaboración son resultado y mérito del equipo y no de alguno de sus integrantes. La agilidad requiere de colaboración, colaboración interna en el equipo y externa con el cliente. Colaborar con el cliente permite guiar de manera regular los resultados del proyecto de desarrollo. O sea que, la colaboración en el marco de agilidad se orienta directamente a conseguir los objetivos del cliente en un proyecto mediante el trabajo en equipo colaborativo. El trabajo en equipo con colaboración del cliente posibilita su frecuente retroalimentación y mantenerse alineado a su punto de vista y sus expectativas para satisfacer sus necesidades o los requisitos del desarrollo, por el cual el sistema producto se desarrolla con mayor agilidad. La colaboración interna requiere soltura y apertura para dejar los egos de lado e integrarse en el proceso de pensamiento colectivo, donde los problemas los resuelven todos los del equipo con sus aportes individuales. En el marco de colaboración se dejan de lados los héroes, pues los héroes no ven el gran dragón Malveau 2004 y los héroes se suelen llevar los créditos. La cooperación es la convicción de que nadie llega a la meta si no llegan todos (Virginia Burden).

\item \textbf{Priorización por valor:} Se prioriza por valor, es decir que se puede ser más efectivo si se hacen primero las tareas que suman más valor al negocio o a las necesidades del cliente. Se hace necesario prescindir de requisitos de baja prioridad antes que tener que degradar la calidad.

\item \textbf{Limitación de tiempos (time-boxing):} El trabajo limitado en periodos de tiempo ayuda en la regularidad en las actividades. Por eso, las iteraciones de trabajo, las actividades y las reuniones deben tener un límite de tiempo y se debe buscar no sobrepasar esos límites.

\item \textbf{Desarrollo iterativo:} El desarrollo iterativo permite una construcción gradual en proyectos complejos. En cada iteración el equipo evoluciona el producto (hace una entrega incremental) a partir de los resultados completados en las iteraciones anteriores, añadiendo nuevos objetivos/requisitos o mejorando los que ya fueron completados, de manera que el cliente pueda obtener los beneficios del proyecto de forma incremental. El desarrollo iterativo permite gestionar las expectativas del cliente (requisitos desarrollados, velocidad de desarrollo, calidad) de manera regular y lograr reacción, aceptación del mercado y adaptabilidad. Permite que el cliente pueda obtener resultados importantes y útiles ya desde las primeras iteraciones. Facilita la mejora ya que al tener experiencias por períodos de iteración se puede mejorar de las experiencias de iteraciones previas y permite planificar los cambios necesarios para aumentar la productividad y calidad en iteraciones subsiguientes.

\end{enumerate}


