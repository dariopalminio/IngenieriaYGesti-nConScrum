%checklists.tex
\newpage
\section{Checklists}

Las Listas de Control o Check Lists son útiles para controlar actividades repetitivas como las ceremonias y lograr un mínimo de control de calidad procedimental. Puede ser de utilidad tener un listado de validación para el inicio de cada ceremonia y otro para el fin. 
A continuación se exponen algunos ejemplos propuestos por el autor:

\subsection{Planning Checklist}

\begin{itemize}
  
  \item {Checklist previo: Antes de la planning se deben corroborar los siguientes incisos:
  
  \begin{itemize}
    \item Invitaciones hechas.
    \item Corroborar asistencias.
    \item Ubicación (lugar para reunirse, disponibilidad de sala).
    \item Recursos materiales necesarios (papel, marcadores, post-its, reloj-cronómetro, cartas de poker-planning, etc.).
    \item Calendario del Sprint a planificar (tener en cuenta feriados, días festivos, capacitaciones, vacaciones, etc.).
    \item Tener en cuenta la capacidad del equipo (datos históricos, velocity y capacity).
    \item El PO tiene un conjunto de historias priorizadas que pasan el DoR para proponer.
    \item El PO o el equipo revisaron el Backlog Técnico para proponer tareas técnicas (bugs, mejoras y deuda técnica).
    \item El PO tiene el objetivo del Sprint para proponer.
    \item El SM debe tener clara la dinámica y programa de la reunión para explicarle al equipo o que el equipo la sepa.
  \end{itemize}
  }
  
  \item {Checklist final: Luego de finalizar la planning se deben corroborar los siguientes incisos:
    \begin{itemize}
      \item Se obtuvo un conjunto de ítems de trabajo para el Sprint Backlog.
      \item Las user stories comprometidas pasaron el DoR.
      \item Se tiene un objetivo de Sprint.
      \item Se hizo el compromiso formal con el PO.
      \item El equipo participó y tiene claros los incisos anteriores.
      \item Si es necesario asentar datos para histórico o métricas (cómo tiempo real insumido, riesgos, etc.).   
   \end{itemize}
  }

\end{itemize}


\subsection{Refinement Checklist}

\begin{itemize}
  
  \item {Checklist previo: Antes del refinamiento se deben corroborar los siguientes incisos:
  
  \begin{itemize}
    \item Invitaciones hechas.
    \item Corroborar asistencias.
    \item Ubicación (lugar y/o disponibilidad de sala).
    \item Recursos materiales necesarios (papel, marcadores, post-its, reloj-cronómetro, cartas de estimación de tamaño, etc.).
    \item El PO tiene un conjunto de historias priorizadas y completas para proponer.
  \end{itemize}
  }
  
  \item {Checklist final: Luego de finalizar el refinamiento se deben corroborar los siguientes incisos:
    \begin{itemize}
      \item Se tiene el backlog actualizado con las historias tratadas.
      \item Si es necesario asentar datos para histórico o métricas (cómo tiempo real insumido, riesgos, etc.).
   \end{itemize}
  }

\end{itemize}

\subsection{Review Checklist}

\begin{itemize}
  
  \item {Checklist previo: Antes de la review se deben corroborar los siguientes incisos:
  
  \begin{itemize}
    \item Invitaciones hechas.
    \item Corroborar asistencias de invitados.
    \item Catering si es necesario (café, galletas, caramelos, etc.).
    \item Ubicación (lugar y/o disponibilidad de sala).
    \item Recursos materiales necesarios (proyector, cables, post-its, marcadores, etc.).
    \item Tener las historias disponibles para presentar.
    \item Las historias deben ser evaluadas anterior a la review por el PO.
    \item Validar cumplimiento del “Definition of Done”.
    \item Presentación (diapositivas) y/o folletería (o afiches) si es necesario.
    \item Tener un programa (Agenda y una dinámica).
    \item Asegurar relevamiento de feedback.
    \item Asegurar ensayo (práctica previa, recolección de evidencias, simulación, etc.).
    \item El PO le comentó a todo el equipo quienes asistirán como Stakeholders.
    \item {Planificación y medición de curso de acción:
      \begin{itemize}
        \item Plan A: Curso normal (software funcionando y disponible).
        \item Plan B: Con imprevistos (plan de contingencia).
        \item Plan C: Cancelación (plan de cancelación).
      \end{itemize}
      }
    \item Disponibilidad de accesibilidad remota (herramienta de video conferencia o comunicación remota).    
      
  \end{itemize}
  }
  
  \item {Checklist final: Luego de finalizar la review se deben corroborar los siguientes incisos:
    \begin{itemize}
      \item Tener claro y asentado qué historias fueron aceptadas y cuáles no (el criterio de aceptación del Sprint Review es que las historias de Usuario sean aceptadas o rechazadas durante la misma).
      \item Tener relevamiento o registro del feedback.
      \item Si es necesario asentar datos para histórico o métricas (cómo métrica de satisfacción de stakeholder, tiempo real insumido, cantidad de Stakeholders, cantidad de feedback, etc.).
   \end{itemize}
  }

\end{itemize}

\subsection{Retrospective Checklist}

\begin{itemize}
  
  \item {Checklist previo: Antes de la retrospectiva se deben corroborar los siguientes incisos:
  
  \begin{itemize}
    \item Invitaciones hechas.
    \item Corroborar asistencias.
    \item Ubicación (lugar y/o disponibilidad de sala).
    \item Recursos materiales necesarios (papel, marcadores, post-its, reloj-cronómetro, etc.).
    \item Se tiene una dinámica a seguir (dinámica clásica, dinámica de la estrella de mar, etc.).
    \item ¿Se tienen datos históricos o acciones de retrospectivas pasadas?

  \end{itemize}
  }
  
  \item {Checklist final: Luego de finalizar la retrospectiva se deben corroborar los siguientes incisos:
    \begin{itemize}
      \item Se tienen identificadas acciones para mejorar.
      \item Si es necesario asentar datos para histórico o métricas (cómo tiempo real insumido, riesgos, etc.).
   \end{itemize}
  }

\end{itemize}
