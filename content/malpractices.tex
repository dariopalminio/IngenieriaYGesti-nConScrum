% Malpractices
% things that are recommended not to do. Why scrum fails ?

\chapter{Malas prácticas}

Velar por la correcta utilización de Scrum es más que el cuidado de un conjunto de reglas a seguir. Existen muchas causas diversas de que se fracase en la implementación de Scrum, además de la de no seguir sus reglas. A esas acciones causantes, resultado de prácticas y que perjudican el buen funcionamiento de Scrum, las podemos llamar malas prácticas y constituirán prácticas que se aconseja evitar.
A continuación se listan algunas de ellas:

\begin{enumerate}

\item \textbf{Aplicar mal la metodología}

Una mala práctica es aplicar mal o en forma incompleta una metodología o técnica determinada. Por ejemplo, se critica a la metodología de cascada (Waterfall) o desarrollo en cascada porque se dice que no funciona, sin embargo lo que suele suceder es que no se aplica realmente (Masa Maeda \footnote{Masa K Maeda es PhD, fundador y CEO de Valueinnova USA (capacitación Scrum). Es el creador de Serious LeAP, consultor senior del Cutter Consortium en Boston, miembro del comité de dirección del Agile Testing Alliance y maestro en la Universidad de California en Berkeley. Es pionero de Lean y Kanban para trabajo de conocimiento y uno de los formadores del Lean Kanban University. Es una figura líder mundial en Agile y ha generado Serious Games de alto calibre.}). El problemas no es que no sirva o no funcione, sino que no lo hacemos bien (Masa Maeda  2012). Pues, en el artículo original de 1970 en el que Royce Winston expone el desarrollo en cascada se habla de ciclos y de “fases sucesivas de desarrollo iterativo” \cite{Winston-Royce-1970} ofreciendo unos consejos a seguir, cosa que no se suele hacer y se da por hecho que cascada no sirve. Algo parecido ocurre con Scrum.

Hay quienes dicen:  "Scrum fracasó". Pero, sin embargo, lo que suele suceder es que se dice que se implementa Scrum aunque, en la práctica, no se cumplen sus recomendaciones o se hacen hibridaciones\footnote{Metodología híbrida o mezcla con Scrum.} con otras metodologías que dan como resultado algo que no es Scrum. Creyendo hacer Scrum, muchas no han logrado superarlo \cite{Gantthead-James-2010}.  

Respecto a este tema en una entrevista que le hicieron a Jeff Sutherland (uno de los creadores de Scrum) él dijo: La mayoría de las empresas implementan Scrum a medias. Por ejemplo, cualquier Scrum sin producto de trabajo al final de un Sprint es un Scrum fracasado y el 80 por ciento del Scrum escalado en Silicon Valley se encuentra en esta categoría, pues son "ágiles sólo de nombre". Cuando una empresa modifica o implementa sólo parcialmente Scrum puede estar ocultando u oscureciendo alguna disfuncionalidad que restringe su competencia en cuanto a gestión y desarrollo de producto (Ken Schwaber 2006).


\item \textbf{Aplicar mal un principio}

En ocasiones en que se aplica mal una metodología se puede deber a mal interpretar sus principios rectores, a generalizarlos o a caer en un una especie de fundamentalismo. Por ejemplo, si consideramos el "trabajo empírico" como algo fundamental, pero basándonos en ello pretendemos que todo trabajo se base en la experiencia directa de los desarrolladores (tipo prueba error) sin recurrir a experiencias previas, a memoria histórica, a procesos organizacionales o a conocimiento teórico. De este modo se puede caer en ser un equipo de trabajo reactivo, ciego de las consecuencias a largo plazo de sus acciones e ineficiente. No es necesario reinventar la rueda cada vez que se nos presenta la necesidad de usar una y a veces eso es lo que ocurre cuando se abusa del trabajo basado en la prueba y el error. Por otro lado, cuando nuestros actos tienen consecuencias que trascienden el horizonte de aprendizaje (nuestra experiencia cercana), se vuelve imposible aprender de la experiencia directa \cite{Peter-Senge-1990}. 
En ocasiones, por poner otro ejemplo, se hace énfasis en el coraje. Pero se puede caer en ser heroico, cuando como dice una frase: "el programador heroico a menudo no ve el gran dragón" (Software Architect Bootcamp). Pues, fomentar el coraje no es fomentar necesariamente las acciones individuales heroicas en vez de buscar sentirse apoyados, trabajar colaborativamente y tener más recursos a disposición para promover el coraje para enfrentar desafíos más grandes.

\item \textbf{No hacer ingeniería por aplicar una metodología}

Metodologías como Scrum  no establecen las prácticas específicas de ingeniería, por lo que se puede aplicar la metodología sin hacer ingeniería. En este caso los Scrum Master son responsables de promover un mayor rigor en la aplicación de las prácticas ingenieriles y de la definición de “terminado” (DoD) acorde al marco de ingeniería \cite{Gantthead-James-2010}. A veces sucede que la agilidad se convierte en un culto, despojando las prácticas reales de ingeniería de software por profesionales ágiles que no tienen una comprensión de la ingeniería de software, y de este modo simplemente se convierte en un conjunto de rituales sin sentido que, en su mayoría, son impedimentos y distracciones a la creación de software con éxito\footnote{\cite{Victor-Hugo-Certuche-2016} \cite{Mike-Hadlow-2014}}.

Desde esta perspectiva, hay que tener en cuenta que el uso de post-its y gráficos bosquejos que parecen infantiles no debería reemplazar el uso de herramientas conceptuales de diagramación como son: Unified Modeling Language, Architecture Description Language, Business Process Modeling Notation, Conceptual Diagram or ConceptDraw, Causal Loop Diagram, Entity Relationship Diagram, Flow Charts (para control de flujo), Data Flow Diagram, Structure Chart, Stock and Flow Diagrams, Structured Systems Analysis and Design Method, Map Mind Diagram, etc; el uso de dinámicas y conversaciones tampoco debería sustituir el Análisis de Sistemas, Investigación Operativa y las prácticas profesionales\footnote{\cite{SWEBOKv3-2014}} de Ingeniería de software; y la simplicidad no debería desplazar el uso de herramientas de software ni la eliminación de métricas fundamentales. La Ingeniería del Software intenta dar un marco de trabajo en el que se aplican práctica del conocimiento científico en el diseño y construcción de software con mayor calidad. 


\item \textbf{No se cambia de mentalidad}

A veces se implementa y usa una nueva metodología pero no se cambia de forma de pensar. Es que se puede considerar que la simple adopción de una herramienta de trabajo, sin una transformación personal que la acompañe, no sirve de nada, por ejemplo adoptar Scrum sin una transformación personal \cite{Martin-Alaimo-Kleer-2014}. 

La gran mayoría de metodologías tienen detrás principios y maneras de pensar. Pues hay que entender que no se trata solo de fórmulas, sino también de formas de razonar. No se puede pretender trabajar en un equipo con alguna metodología ágil que implementa auto-organización si no se cree en la auto-organización. Hay personas que creen en el liderazgo centralizado y autoritario, por lo que no cambian su perspectiva y, en vez de adaptarse a la nueva manera de trabajar, terminan queriendo adaptar la manera de trabajar a su idea original, por ejemplo a la conducción centralizada y autoritaria. Esta actitud termina por generar malas prácticas que socavan el buen funcionamiento de una metodología determinada.

\item \textbf{No se cambia realmente la manera de trabajar}

Existen aquellos casos en donde no se cambia realmente la manera de trabajar y se aplica un “Scrum cosmético”\footnote{Un modelo de madurez para equipos ágiles, Angel Medinilla, Jan 15 2015.}. Esto quiere decir que se agregan las ceremonias y los roles necesarios, pero realmente no se aplica Scrum ni se abandona la forma de trabajar que se venía haciendo hasta el momento.

\item \textbf{Disociación entre producción y resto de la organización}

La disociación entre producción y resto de la organización se da cuando se implementa agilidad, como Scrum, solo en el área de producción para hacer organizar el proceso de desarrollo, pero la gestión estratégica o las capas de gestión de alto nivel de la compañía desconocen la filosofía ágil o Scrum y gestionan los portfolios, programas y proyectos con las metodologías criticadas en este marco de trabajo. Algo semejante sucede con los vendedores de la organización. Pues, si los mismos venden productos especificados de antemano y en base a esas ventas se realizan compromisos contractuales rígidos y se exige que el área de producción cumpla con esos compromisos, por más que el área de producción intente usar Scrum, se puede caer en los mismos problemas que con Scrum se critica e incumplir con el o los proyectos. El uso de Scrum para lograr una organización de gestión y de desarrollo de un producto optimizado, es un proceso de cambio que debe ser dirigido o acompañado por las altas esferas de la compañía y que requiere que todos en la organización hagan cambios en sintonía (Ken Schwaber 2006).

\item \textbf{Disociación entre Desarrollo y Operaciones}

Si Desarrollo es ágil pero Operaciones no, tendremos un gran impedimento para lograr un DevOps eficiente y entrega contínua \footnote{Disociación entre el desarrollo y el resto de la organización se la conoce como Water-SCRUM-fall; y que significa, desde el punto de vista de DevOps, que mientras los equipos de desarrollo pueden haber adoptado prácticas ágiles, los equipos de operaciones no \cite{DevOps-for-dummies-2015}.}. Operaciones pueden generar bloqueos y cuellos de botella debido a: procesos pesados que generan burocracia, falta de personal para dar soporte, personal con poca idoneidad, cultura waterfall, mala comunicación con los equipos de desarrollo, desincronización de mantenimientos o tareas técnicas y desarrollo que pueden provocar bloqueos generales, escaso o falta de soporte IT, infraestructura con mal funcionamiento, etcétera.

\item \textbf{Equipos con integrantes aislados}

Una mala costumbre en el trabajo en equipos es el trabajo en forma aislada de los integrantes, aún estando en la misma oficina. Ejemplos pueden ser las reuniones de daily hechas por alguna herramienta informática de conferencia (como Google Hangouts, Skype, etc.), el trabajo de integrantes ermitaños sentados solos y con auriculares o el trabajo de equipos donde todos sus integrantes son remotos. Esto debe evitarse priorizando el trabajo codo a codo, cara a cara.

\item \textbf{Las reuniones como fin}

Hay que tener en cuenta que las reuniones son un medio y no un fin. A veces se cae en una cultura de reuniones y minutas, como si se tratara de un "vicio organizacional"\footnote{\cite{UNTREF-2014}}. Por otro lado, en las organizaciones donde prima la confianza no es necesario asentar toda reunión en minutas. Las minutas deberían ser recordatorios y no acuerdos contractuales.

\item \textbf{Excesivo foco en la entrega}

El foco en la entrega, en la industria de software, es un anti-patrón que se ha arrastrado por años y que se consideraba como la forma de trabajo eficiente\footnote{\cite{Erich-Buhler-2015}}. El foco en la entrega se refiere a cuando los equipos se centran en entregar, el éxito se mide en la cantidad de características que se entrega y en consecuencia hay una presión de cumplir con determinadas entregas pactadas o con determinadas características comprometidas. Cuando aumenta el foco en la entrega ocurre que disminuye el aprendizaje del equipo, la creatividad, la innovación y posiblemente el posicionamiento. También ocurre que se gatillan sistemas de control mediante la solicitud de informes, reportes o presión social. Todo esto lleva un ciclo vicioso de trabajo estresante.

Para vencer el foco en la entrega hay que equilibrar con el aprendizaje. La agilidad impulsa que el enfoque debe ser el aprendizaje, que todos hayan aprendido nuevas y mejores formas de hacer las cosas, que el conocimiento se distribuya libremente y que los requisitos dejen de ser requerimientos y sean hipótesis a convalidar o invalidar, lo que requerirá conocimiento y maduración continua\footnote{\cite{Erich-Buhler-2015}}.

\item \textbf{Estandarización en base a medidas subjetivas}

En algunas organizaciones se incurre en una simplificación mecánica, fuera del marco ágil, cuando se insiste en comparar a los equipos usando SP, medir sus velocidades como indicativo de productividad y estandarizar niveles de madurez basados en ella. 
Desconocer que los SP de historia son una unidad de medida relativa y subjetiva para expresar una estimación del esfuerzo, incertidumbre y/o complejidad, no reconocer que es tan relativa que su tamaño varía en el tiempo según la subjetividad de quien los determinan y que los SP de un equipo pueden ser totalmente distintos a los de otro; es un indicio de no entender la agilidad. A veces el deseo y la necesidad de control y maximización de producción nubla la concepción de que la industria de software se basa en el trabajo intelectual y creativo, sobre un producto de contenido prácticamente intangible, como es el software. Si bien es necesario medir, controlar y planificar, siempre hay que tener en cuenta que la industria de software no es una manufactura de trabajo repetitivo, mecánico y en serie (producción en cadena). Por tal motivo hay que prestar particular importancia a la forma en que se mide la productividad y eficiencia; y en cómo se comparan equipos.

\item \textbf{Se hace lo que el Mesías dice}

La mala práctica de “se hace lo que el Mesías dice” se refiere al caso en que un líder, un conjunto de líderes o una parte de la organización se comportan como un rey monarca que da saltos de fe hacia un consejero, cual si fuera un Mesías. Un mesías pueden ser algún Agile Coach consultor o una empresa de consultoría Ágile. Pues, que alguien sea Agile Coach no significa que tiene la bala de plata, que entienda de sistemas o que tiene la solución al problema o cambio organizacional que estamos necesitando. A veces sucede que los Agile Coaches son defensores de lo que saben (Scrum, Lean, Kanban, SAFe, LeSS, Crystal u otra metodología o técnica) y venden soluciones empaquetadas que no son necesariamente la solución óptima. Cada organización es un mundo y tiene sus particularidades que se deben analizar. Ninguna metodología resuelve todos los problemas y es necesario integrar diferentes, según el contexto y estado organizacional, según la propia experiencia organizacional y con herramientas de ingeniería de sistemas.


\end{enumerate}
