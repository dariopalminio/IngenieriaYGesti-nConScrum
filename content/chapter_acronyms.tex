\chapter{Glosario y Acrónimos}


  \begin{description}    
  
  \item {\textbf{Agile:} Agile, agilismo, ágil o agilidad se refiere -en general- al lineamiento con los valores ágiles expresados en el manifiesto ágil, cimientos que son cualidades que consideramos valiosas o deseables tener en cuenta.}
   
  \item {\textbf{Artifact:} Artefacto o almacén de incisos de trabajo.}  
 
  \item {\textbf{BDUF:} Gran diseño al inicio (Big Design Up Front).}  
  
  \item {\textbf{BA:} Analista de Negocio (Business Analyst). Un BA es quien realiza tareas de análisis de negocio que se describen en BABOK Guide. El BA es responsables de descubrir, analizar y sintetizar la información de una variedad de fuentes dentro de una empresa, incluyendo herramientas, procesos, documentación, y los stakeholders. Es responsable del relevamiento de las necesidades reales de los stakeholders, lo cual con frecuencia involucra la investigar y aclarar sus deseos expresados con el fin de determinar los problemas y las causas subyacentes en el dominio del negocio. Ellos juegan un papel importante en la adaptación de las soluciones diseñadas y entregadas según las necesidades de los stakeholders. Las actividades que realizan incluyen: comprensión de las metas de la empresa y sus problemas, análisis de necesidades y soluciones, análisis de la organización (estructura, política y operaciones), la elaboración de estrategias, impulsión del cambio, y facilitación de la colaboración de los stakeholders.}
   
  \item {\textbf{BE:} Software de bajo nivel o detrás de la interfaz de usuario (Back-End). Su desarrollo especializado sule darse por BE developer (BE Dev).}
 
  \item {\textbf{DoD:} Definición de terminado (Definition of Done).}
  
  \item {\textbf{DoR:} Definición de completitud (Definition of ready).}
  
  \item {\textbf{DT:} Equipo de desarrollo (Development Team).}
  
  \item {\textbf{DUF:} Diseño al inicio (Design Up Front).}

  \item {\textbf{FE:} Software de Interfaz de Usuario (Front-End). Su desarrollo especializado sule darse por FE developer (FE Dev), desarrolladores de interfaz gráfica (Dev UI) y diseñadores UI.}
  
  \item {\textbf{Features:} Representan interacciones y acciones del usuario con el sistema. Son funcionalidades que entregan valor de cara al usuario.}
  
  \item {\textbf{MMP:} Producto comerciable mínimo (Minimal Marketable Product).}
  
  \item {\textbf{MVP:} Producto viable mínimo (Minimal Viable Product).}
  
  \item {\textbf{PB:} Backlog de producto (Product Backlog).}
  
  \item {\textbf{PBI:} Ítem de Product Backlog Item (story, technical story, technical debt, spike, etc.).}
  
  \item {\textbf{PI:} Incremento de producto (Product Increment).}
  
  \item {\textbf{PMI:} Instituto de Gestión de Proyectos (Project Management Institute).}
  
  \item {\textbf{QA:} Rol específico para hacer pruebas de software y asegurar la calidad (Quality Assurance Tester).}
  
  \item {\textbf{Risk:} Riesgo es la posibilidad de un problema o incertidumbre relacionada con un proyecto o PBI de un proyecto, que podría alterar significativamente el resultado del mismo de una manera potencialmente negativa. No tiene ningún impacto actual en el proyecto, pero podría tener un impacto potencial en el futuro.}
  
  \item {\textbf{ROI:} Retorno de la inversión (Return On Investment).}
  
  \item {\textbf{SCRUM:} Es un marco de trabajo. El término fue tomado prestado del rugby. En rugby es un juego en el que, por lo general, tres miembros de cada línea se unen opuestos unos a otros con un grupo de dos y un grupo de tres jugadores detrás de ellos, lo que hace un grupo de ocho personas, tres, dos, tres formados en cada lado; el balón se deja entre la línea divisoria de ambos grupos, los jugadores están abrazados y tomados de la cintura de un compañero de equipo y los del frente hombro a hombro con el oponente, y se trata hacer fuerza grupalmente para desplazar al grupo rival y patear la pelota hacia atrás para que un compañero de equipo la tome.}
  
  \item {\textbf{SB:} Backlog de iteración (Sprint Backlog).}
  
  \item {\textbf{Scaling Scrum:} Es cualquier implementación de Scrum donde múltiples Equipos Scrum construyen un producto, múltiples productos relacionadoso o un conjunto de características de un producto en uno o más Sprints (NEXUS, Scrum Profesional a Escala, Lucho Salazar, Versión 3.0.0,  Agiles 2016 en Quito, 6-8 Octubre, 2016).}
  
  \item {\textbf{SM:} Facilitador (ScrumMaster).}
  
  \item {\textbf{SOA:} Arquitectura Orientada a Servicios (Service-Oriented Architecture).}
  
  \item {\textbf{Spt:} Iteración (Sprint).}
  
  \item {\textbf{SP:} Story Point o Sp.}
  
  \item {\textbf{UX:} Se refiere a un rol encargado del análisis y desarrollo de la experiencia de usuario de un sistema (User Experience).}
  
\end{description}
  
