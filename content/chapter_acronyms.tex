\chapter{Glosario y Acrónimos}


  \begin{description}  
  
  \item {\textbf{BDUF:} Gran diseño al inicio (Big Design Up Front).}
  
  \item {\textbf{Artifact:} Artefacto o almacén de incisos de trabajo.}  
  
  \item {\textbf{DoD:} Definición de terminado (Definition of Done).}
  
  \item {\textbf{DoR:} Definición de completitud (Definition of ready).}
  
  \item {\textbf{DT:} Equipo de desarrollo (Development Team).}
  
  \item {\textbf{DUF:} Diseño al inicio (Design Up Front).}

  \item {\textbf{Features:} Representan interacciones y acciones del usuario con el sistema. Son funcionalidades que entregan valor de cara al usuario.}
  
  \item {\textbf{MMP:} Producto comerciable mínimo (Minimal Marketable Product).}
  
  \item {\textbf{MVP:} Producto viable mínimo (Minimal Viable Product).}
  
  \item {\textbf{PB:} Backlog de producto (Product Backlog).}
  
  \item {\textbf{PBI:} Ítem de PB (Product Backlog Item).}
  
  \item {\textbf{PI:} Incremento de producto (Product Increment).}
  
  \item {\textbf{PMI:} Instituto de Gestión de Proyectos (Project Management Institute).}
  
  \item {\textbf{ROI:} Retorno de la inversión (Return On Investment).}
  
  \item {\textbf{SCRUM:} Es un juego de rugby en el que, por lo general, tres miembros de cada línea se unen opuestos unos a otros con un grupo de dos y un grupo de tres jugadores detrás de ellos, lo que hace un grupo de ocho personas, tres, dos, tres formados en cada lado; el balón se deja entre la línea divisoria de ambos grupos, los jugadores están abrazados y tomados de la cintura de un compañero de equipo y los del frente hombro a hombro con el oponente, y se trata hacer fuerza grupalmente para desplazar al grupo rival y patear la pelota hacia atrás para que un compañero de equipo la tome.}
  
  \item {\textbf{SB:} Backlog de iteración (Sprint Backlog).}
  
  \item {\textbf{SM:} Facilitador (ScrumMaster).}
  
  \item {\textbf{SOA:} Arquitectura Orientada a Servicios (Service-Oriented Architecture).}
  
  \item {\textbf{Spt:} Iteración (Sprint).}
  
  \item {\textbf{SP:} Story Point o Sp.}
  
\end{description}
  
