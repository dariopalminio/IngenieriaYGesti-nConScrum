% Acknowledgments page and Preface section
% Pages before of the index

%---------------------------------------------------------------------------------------------
% Acknowledgments page (Página de agradecimientos)
%

\newpage

% White spaces
\hspace{1cm}\newline %1cm horizontal space
\vspace{5cm} %5cm vertical space

\begin{center}
Gracias a todos los hombres y mujeres que van a trabajar cada día y llevan con ellos su calor humano; contribuyendo positivamente a su equipo, a su organización y al mundo por ser lo que son, haciendo lo que hacen, cuando lo hacen.
\end{center}

% Empty White Page
\newpage
\vspace{2cm} %2cm vertical space
\hspace{1cm}\newline %1cm horizontal space

%---------------------------------------------------------------------------------------------
% Preface Page
%

\newpage

{\large \textbf{Prefacio}}

\vspace{2cm} %2cm vertical space

Quien escribe se propuso ofrecer una guía para el conocimiento e implementación de una manera de trabajar y de gestionar proyectos de Ingeniería de Software llamada Scrum. Se propone explicarle al lector el sistema Scrum de un modo integrador desde diferentes perspectivas y fuentes de conocimiento Scrum. De este modo se intenta proporcionar un marco integral que incluya los principios filosóficos, estructura y procesos de Scrum y aspectos complementarios. Se busca ayudar a los equipos a avanzar de intentar emplear Scrum a ejecutar Scrum correctamente para lograr alcanzar los resultados que aún no se han alcanzado. También se pretende brindar un material de apoyo para: facilitadores que emprendan el coaching de equipos, equipos de desarrollo de software e interesados en la Ingeniería de Software.

% Empty White Page
\newpage
\vspace{2cm} %2cm vertical space
\hspace{1cm}\newline %1cm horizontal space
 
